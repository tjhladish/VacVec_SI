\documentclass{article}
\usepackage[utf8]{inputenc}
\usepackage{hyperref}

\title{Synthetic Population Yucatan - Pipeline}
\author{}
\date{September 2018}

\begin{document}

\section{Data}
\begin{itemize}
\item Yucatan schools information
    \begin{itemize}
        \item Source:
        \item Description: Data table containing names, addresses, localities and municipalities of 3290
        schools
        \item Usage: To generate schools coordinates
    \end{itemize}
\item Nighttime light raster - VIIRS Day/Night Band Nighttime Lights
    \begin{itemize}
        \item Source: Earth Observation Group, NOAA \url{https://ngdc.noaa.gov/eog/viirs/download_dnb_composites.html}
        \item Description: Annual composites of the Visible Infrared Imaging Radiometer Suite (VIIRS) Day/Night Band (DNB) radiance in 2015. Data provider has removed the outliers, screened out the ephemeral lights and set the background (non-lights) to zero.
        \item Usage: As a proxy to human activities and population -- Schools are more likely to locate at places with higher nighttime light. 
    \end{itemize}
\item Localities and municipalities shapefiles
    \begin{itemize}
        \item Source: Mexican population and housing census 2010, provided by Instituto Nacional de Estadística y Geografía (INEGI) via Sistema para la Consulta de Información Censal (SCINCE)
        \item Description: INEGI has provided shapefiles of the 106 municipalities of Yucatan and the localities within them. For urban localities, polygons are provided; for rural localities, only points are provided.
        \item Usage: Random allocation of schools coordinates -- localities and municipalities of schools that cannot be geocoded are matched with these shapefiles.
        \item Additional references:  \href{http://www.inegi.org.mx/est/scince/scince2010.aspx?_file=/est/scince/scince2010/Scince2010_31.exe&idusr=80085%22%20-o%20yuc_scince.exe} {Download links (in .exe)}
        and \href{https://blog.diegovalle.net/2013/06/shapefiles-of-mexico-agebs-manzanas-etc.html}{more information.}
    \end{itemize}
\end{itemize}

\section{School coordinates}
The geographic coordinates for the 3290 schools were determined using following steps:
\begin{enumerate}
    \item We use the Google Map's Geocoding API to convert the school addresses to coordinates. 1374 schools could be geocoded.
    \item For the rest of the 1916 schools, we match each of their locality and municipality with the SCINCE shapefiles.
    \item There were 1867 schools with matching polygons (urban) or points (rural), 53 of which had more than one match. The following steps were taken:
    \begin{itemize}
        \item For each urban polygon, all pixels within the polygon were selected.
        \item For each rural points, all pixels (a) within the 5km distance from the point, (b) within the municipality, and (c) not belong to an urban polygon were selected.
        \item Randomly sample one out of all selected pixels, with probability of selection weighted by the pixels' nighttime light value.
        \item Assign the coordinates of the pixel to the school.
    \end{itemize}
    \item For the 49 schools of which locality could not be matched with the shapefiles:
    \begin{itemize}
        \item Select all pixels within the municipality's polygons that do not belong to an urban polygon (i.e. assuming the school is not in urban area).
        \item Randomly sample one out of all selected pixels, with probability of selection weighted by the pixels' nighttime light value.
        \item Assign the coordinates of the pixel to the school.
    \end{itemize}
\end{enumerate}  
\end{document}
